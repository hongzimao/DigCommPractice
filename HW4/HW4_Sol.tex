\documentclass [12pt] {article}
\usepackage[top=1in, bottom=0.9in, left=0.9in, right=0.9in]{geometry}
\usepackage{graphicx}
\usepackage{setspace}
\usepackage{hyperref}
\usepackage{listings}
\usepackage{courier}
\usepackage[export]{adjustbox}
\usepackage[english]{babel}
\usepackage[]{algorithm2e}
\usepackage{amsmath}
\usepackage{mathrsfs}
\usepackage{mathtools}
\usepackage{amsfonts}
\makeatletter
\newcommand*{\myfnsymbolsingle}[1]{%
  \ensuremath{%
    \ifcase#1% 0
    \or % 1
    \dagger
    \or % 2
     \dagger 
    \or % 3  
     \dagger
    \or % 4   
      \mathsection
    \or % 5
      \mathparagraph
    \else % >= 6
      \@ctrerr  
    \fi
  }%   
}   
\makeatother

\newcommand*{\myfnsymbol}[1]{%
  \myfnsymbolsingle{\value{#1}}%
}

% remove upper boundary by multiplying the symbols if needed
\usepackage{alphalph}
\newalphalph{\myfnsymbolmult}[mult]{\myfnsymbolsingle}{}

\renewcommand*{\thefootnote}{%
  \myfnsymbolmult{\value{footnote}}%
}

\begin {document}
\begin{spacing}{1.5}
\begin{center}
\small
\textbf{ELEC 5360 Homework 4}\\ Hongzi Mao 2002 5238\\
\end{center}
\normalsize
~\\
1. \\
The log-likelihood function is given by 
\begin{align*}
 \Lambda _L (\tau) &= \Re[\frac{1}{N_0} \int_{T_0} r(t) s_l^*(t,\tau) dt]\\
&=\Re[\frac{1}{N_0} \int_{T_0} r(t) \sum_{n} I_n g^*(t - nT -\tau) dt]\\
&=\Re[\sum_n I_n y_n(\tau)]
 \end{align*}
 where $ y_n = \frac{1}{N_0} \int_{T_0} r(t)g^*(t - nT -\tau) dt$\\
 In order to maximize the likelihood function, we set $$ \frac{d \Lambda _L (\tau)}{d\tau} = \Re[\sum_n I_n \frac{d}{d\tau}y_n(\tau)] = 0$$
 Further simply the expression, $$\Re[ \sum_n I_n \frac{d}{d\tau}y_n(\tau)] = \sum_n\Re[I_n]\Re[\frac{d}{d\tau}y_n(\tau)] - \sum_n\Im[I_n]\Im[\frac{d}{d\tau}y_n(\tau)]$$
 Therefore we obtain, $$ \sum_n\Re[I_n]\Re[\frac{d}{d\tau}y_n(\tau)] = \sum_n\Im[I_n]\Im[\frac{d}{d\tau}y_n(\tau)]$$
~\\
\pagebreak
~\\
2. (a) \\
\begin{center}
\includegraphics[width=0.6\textwidth]{3a.jpg}\\
\end{center}
(b) \\
The code rate is 1/3. \\
The constrain length is 3. \\
~\\
(c)\\
\begin{center}
\includegraphics[width=0.6\textwidth]{3c.jpg}\\
\end{center}
(d)\\
Split the state a to \emph{start} state a and \emph{end} state e as shown below. \\
\begin{center}
\includegraphics[width=0.6\textwidth]{3d.jpg}\\
\end{center}
~\\
Then we obtain
\begin{align*}
X_b &= D^2 X_a + D^2 X_c\\
X_c &= D X_b + D X_d \\
X_d &= D^3 X_b + D X_d \\
X_e &= D^2 X_c\\
\end{align*}
Simplify the set of equations, we will get 
\begin{align*}
X_e &= D^2(D X_b +D X_d) \\
X_d &= D^3/(1-D) X_b \\
D^2 X_a &= X_b - D^2(D X_b + D X_d)
\end{align*}
Then 
\begin{align*}
\frac{X_e}{X_a} &= \frac{(D^3+D^6/(1-D))D^2}{1-D^3 + D^6/(1-D)}\\
&= \frac{D^5(1-D) + D^8}{(1-D^3)(1-D)+D^6}\\
&= \frac{D^5-D^6+D^8}{1-D-D^3+D^4+D^6}
\end{align*}
(e) \\
From last equation, we obtain 
$$\frac{X_e}{X_a} = (D^5 -D^6 + D^8) [1+(D+D^3-D^4-D^6) + (D+D^3-D^4-D^6)^2 + ...] $$
Therefore the minimum free distance $d_{free}$ of the code is 5.\\
~\\
3. \\
Notice that $f_k - f_j = n/T$ for $n = 1,2,..., N-1$ (as $f_k \neq f_j$), also under the assumption that $f_k+f_j \gg 1$, Then
\begin{align*}
&\int_0^T cos(2\pi f_k t + \phi_k)cos(2\pi f_j t +\phi_j) dt\\
&=\frac{1}{2}\int_0^T cos(2\pi f_k t + \phi_k + 2\pi f_j t +\phi_j) dt +\frac{1}{2} \int_0^T cos(2\pi f_k t + \phi_k - 2\pi f_j t -\phi_j) dt\\
&=\frac{1}{2}\int_0^T cos(2\pi (f_k + f_j) t + \phi_k +\phi_j) dt +\frac{1}{2} \int_0^T cos(2\pi nt/T + \phi_k -\phi_j) dt\\
&=\frac{1}{2\pi\frac{n}{T}} \frac{1}{2} sin(2\pi\frac{nt}{T} + \phi_k - \phi_j) \big|_0^T + \frac{1}{2\pi(f_k + f_j)}\frac{1}{2} sin(2\pi (f_k + f_j) t + \phi_k +\phi_j)\big|_0^T
\end{align*}
Notice that $$ sin(2\pi\frac{nt}{T} + \phi_k - \phi_j) \big|_0^T  = sin(2\pi n + \phi_k - \phi_j) - sin( \phi_k - \phi_j) = 0$$
and $$ \frac{1}{2\pi(f_k + f_j)} \simeq 0 \text{ for large } f_k+f_j$$
Hence we get $\int_0^T cos(2\pi f_k t + \phi_k)cos(2\pi f_j t +\phi_j) dt = 0$ and therefore subcarriers of the corresponding sub-channels are mutually orthogonal.\\
~\\
4. (a)\\
If we dont account the cyclic prefix, the maximum symbol rate for each subcarrier relates to the $\Delta f$. So the ideal maximum symbol rate will be 15k symbol/sec. However, in this specific case, taking the CP in to consideration, using the least amount of CP, we get the maximum symbol rate $ (\text{size}(FFT) - \text{size}(CP))/(\text{size}(FFT)\times T_{sample}) = (2048 - 144)/(2048/(\Delta f \times 2048)) =  13.95 \text{k symbol/sec}$\\
~\\
(b)\\
The symbol duration is given by $T = N T_{tb}$, where $T_tb = 1/f_{tb} $ and $N = 20\text{ MHz} / 15\text{ kHz}$.
Plugging the numbers in, we get $ 20\text{ MHz} / 15\text{ kHz} / 20\text{ MHz} = 1/15 \approx 0.0667 \text{ sec}$.\\
~\\
(c)\\
One view of the need for CP in OFDM is, in the guarding interval preventing the inter-OFDM system-interference, we may as well arising something to further prevent information loss. More formally, in the channel matrix point of view, cyclic prefix will lead to a circular matrix which can further leads to diagonal matrix decomposition, this help simplify the matrix a lot. \\
~\\
(d) \\
The power loss can be defined as the ratio between redundant information propagated and symbol transmitted. \\
Therefore, for normal CP, we get $ 144/2048 $, for extended CP, $512/2048$\\
~\\
(e) \\
We need to require the guarding interval spread wider than the delay, namely $CP/(\text{size}(FFT)\Delta f) \geq 7\mu s$, where $\Delta f = 15\text{ kHz}$. Then we get $CP \geq 215.04$. Therefore the extended 512 CP is needed.\\
\end{spacing}
\end {document}
