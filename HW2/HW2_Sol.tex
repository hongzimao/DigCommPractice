\documentclass [12pt] {article}
\usepackage[top=1in, bottom=0.9in, left=0.9in, right=0.9in]{geometry}
\usepackage{graphicx}
\usepackage{setspace}
\usepackage{hyperref}
\usepackage{listings}
\usepackage{courier}
\usepackage[export]{adjustbox}
\usepackage[english]{babel}
\usepackage[]{algorithm2e}
\usepackage{amsmath}
\usepackage{mathrsfs}
\usepackage{mathtools}
\usepackage{amsfonts}
\makeatletter
\newcommand*{\myfnsymbolsingle}[1]{%
  \ensuremath{%
    \ifcase#1% 0
    \or % 1
    \dagger
    \or % 2
     \dagger 
    \or % 3  
     \dagger
    \or % 4   
      \mathsection
    \or % 5
      \mathparagraph
    \else % >= 6
      \@ctrerr  
    \fi
  }%   
}   
\makeatother

\newcommand*{\myfnsymbol}[1]{%
  \myfnsymbolsingle{\value{#1}}%
}

% remove upper boundary by multiplying the symbols if needed
\usepackage{alphalph}
\newalphalph{\myfnsymbolmult}[mult]{\myfnsymbolsingle}{}

\renewcommand*{\thefootnote}{%
  \myfnsymbolmult{\value{footnote}}%
}

\begin {document}
\begin{spacing}{1.5}
\begin{center}
\small
\textbf{ELEC 5360 Homework 2}\\ Hongzi Mao 2002 5238\\
\end{center}
\normalsize
~\\
1. \\
The initial value of $a_{i}$ determines $b_{i}$ as follows:
\begin{align*}
b_1 & = \frac{a_1+a_2}{2} = 0.4\\
b_2 & = \frac{a_2+a_3}{2} = 0.65\\
b_3 & = \frac{a_3+a_4}{2} = 0.8
\end{align*}
According to \emph{Lloyd-Max Algorithm},
\begin{align*}
a_1 & = \frac{\int_{b_0}^{b_1} x f_X(x) dx}{\int_{b_0}^{b_1} f_X(x) dx} = \frac{\int_{0}^{0.4} x (-2x+2) dx}{\int_{0}^{0.4} -2x+2 \;dx} = \frac{-2/3 x^3 +x^2 \Big|_0^{0.4}}{-x^2 + 2x \Big|_0^{0.4}} = \frac{11}{60}\\
a_2 & = \frac{\int_{b_1}^{b_2} x f_X(x) dx}{\int_{b_1}^{b_2} f_X(x) dx} = \frac{\int_{0.4}^{0.65} x (-2x+2) dx}{\int_{0.4}^{0.65} -2x+2 \;dx} = \frac{-2/3 x^3 +x^2 \Big|_{0.4}^{0.65}}{-x^2 + 2x \Big|_{0.4}^{0.65}} = \frac{293}{570}\\
a_3 & = \frac{\int_{b_2}^{b_3} x f_X(x) dx}{\int_{b_2}^{b_3} f_X(x) dx} = \frac{\int_{0.65}^{0.8} x (-2x+2) dx}{\int_{0.65}^{0.8} -2x+2 \;dx} = \frac{-2/3 x^3 +x^2 \Big|_{0.65}^{0.8}}{-x^2 + 2x \Big|_{0.65}^{0.8}} = \frac{79}{110}\\
a_4 & = \frac{\int_{b_3}^{b_4} x f_X(x) dx}{\int_{b_3}^{b_4} f_X(x) dx} = \frac{\int_{0.8}^{1} x (-2x+2) dx}{\int_{0.85}^{1} -2x+2 \;dx} = \frac{-2/3 x^3 +x^2 \Big|_{0.8}^{1}}{-x^2 + 2x \Big|_{0.8}^{1}} = \frac{13}{15}
\end{align*}
New $b_{i}$ can be then calculated accordingly,
\begin{align*}
b_1 & = \frac{a_1+a_2}{2} = \frac{53}{152}\\
b_2 & = \frac{a_2+a_3}{2} = \frac{3863}{6270}\\
b_3 & = \frac{a_3+a_4}{2} = \frac{523}{600}
\end{align*}
~\\
2. (a)\\
To check the orthogonality, we look into
 \begin{align*}
\int_{0}^{4} f_{1}(t)f_{2}^{*}(t) dt & = \int_{0}^{2} \frac{1}{2}\frac{1}{2} dt + \int_{2}^{4} \frac{1}{2}(-\frac{1}{2}) dt = -\frac{2}{4} + \frac{2}{4} = 0\\
\int_{0}^{4} f_{1}(t)f_{3}^{*}(t) dt & = \int_{0}^{1} \frac{1}{2}\frac{1}{2} dt + \int_{1}^{2} \frac{1}{2}(-\frac{1}{2}) dt + \int_{2}^{3} \frac{1}{2}\frac{1}{2} dt  +\int_{3}^{4} \frac{1}{2}(-\frac{1}{2}) dt = \frac{1}{4} - \frac{1}{4} +  \frac{1}{4} - \frac{1}{4} = 0\\
\int_{0}^{4} f_{1}(t)f_{3}^{*}(t) dt & = \int_{0}^{1} \frac{1}{2}\frac{1}{2} dt + \int_{1}^{2} \frac{1}{2}(-\frac{1}{2}) dt + \int_{2}^{3} (-\frac{1}{2})\frac{1}{2} dt  +\int_{3}^{4} (-\frac{1}{2})(-\frac{1}{2}) dt = \frac{1}{4} - \frac{1}{4} -  \frac{1}{4} + \frac{1}{4} = 0
\end{align*}
To check the normality, we look into
 \begin{align*}
\int_{0}^{4} f_{1}(t)f_{1}^{*}(t) dt & = \int_{0}^{4} \frac{1}{2}\frac{1}{2} dt  = \frac{4}{4} = 1\\
\int_{0}^{4} f_{2}(t)f_{2}^{*}(t) dt & = \int_{0}^{2} \frac{1}{2}\frac{1}{2} dt + \int_{2}^{4} (-\frac{1}{2})(-\frac{1}{2}) dt  = \frac{2}{4} + \frac{2}{4} = 1\\
\int_{0}^{4} f_{3}(t)f_{3}^{*}(t) dt & = \int_{0}^{1} \frac{1}{2}\frac{1}{2} dt + \int_{1}^{2} (-\frac{1}{2})(-\frac{1}{2}) dt + \int_{2}^{3} \frac{1}{2}\frac{1}{2} dt  +\int_{3}^{4} (-\frac{1}{2})(-\frac{1}{2}) dt = \frac{1}{4} + \frac{1}{4} +  \frac{1}{4} + \frac{1}{4} = 1
\end{align*}
Hence the bases are orthonormal to each other. \\
~\\
(b)\\
Assume $x(t)$ can be decomposed into the set of bases defined above, then
$$ x(t) = a_1 f_1(t) + a_2 f_2(t) + a_3 f_3(t) $$
Now
 \begin{align*}
a_1 = \int_{0}^{4} x(t)f_{1}^{*}(t) dt & = \int_{0}^{1} (-1)\frac{1}{2} dt + \int_{1}^{3} 1\frac{1}{2} dt  + \int_{3}^{4} (-1)\frac{1}{2} dt  = -\frac{1}{2} + \frac{2}{2} - \frac{1}{2} = 0\\
a_2 = \int_{0}^{4} x(t)f_{2}^{*}(t) dt & = \int_{0}^{1} (-1)\frac{1}{2} dt + \int_{1}^{2} 1\frac{1}{2} dt  + \int_{2}^{3} 1(-\frac{1}{2}) dt + \int_{3}^{4} (-1)(-\frac{1}{2}) dt  = -\frac{1}{2} + \frac{1}{2} - \frac{1}{2} + \frac{1}{2} = 0\\
a_3 = \int_{0}^{4} x(t)f_{3}^{*}(t) dt & = \int_{0}^{1} (-1)\frac{1}{2} dt + \int_{1}^{2} 1(-\frac{1}{2}) dt  + \int_{2}^{3} 1\frac{1}{2} dt + \int_{3}^{4} (-1)(-\frac{1}{2}) dt  = -\frac{1}{2} - \frac{1}{2} + \frac{1}{2} + \frac{1}{2} = 0
\end{align*}
All the coefficients are 0, which means the signal doesn't live in the space span by the bases. \\
~\\
3. (a)\\
According to \emph{Gram-Schemidt method}, denote $m_{i}$ to be the waveform after removing components of constructed bases, and $x_{i}$ to be the normalized waveform basis. 
\begin{align*}
m_1(t) &= s_1(t)\\
x_1(t) &= m_1(t) * \frac{1}{\int_0^T m_1(t)dt} =  m_1(t) * \frac{1}{\int_0^{T/3} 1 \;dt} = \frac{3}{T} s_1(t)\\ 
m_2(t) &= s_2(t)-[\int_0^T s_2(t)x_{1}^{*}(t)]x_1(t) = s_2(t) - \frac{T}{3} x_1(t)\\
x_2(t) &= m_2(t) * \frac{1}{\int_0^T m_2(t)dt} =  m_1(t) * \frac{1}{\int_{T/3}^{2T/3} 1 \;dt} = \frac{3}{T}( s_2(t) - s_1(t))\\ 
m_3(t) &= s_3(t)-[\int_0^T s_3(t)x_{1}^{*}(t)]x_1(t)-[\int_0^T s_3(t)x_{2}^{*}(t)]x_2(t) = s_3(t) - \frac{T}{3} x_2(t)\\
x_3(t) &= m_3(t) * \frac{1}{\int_0^T m_3(t)dt} =  m_3(t) * \frac{1}{\int_{2T/3}^{T} 1 \;dt} = \frac{3}{T}( s_3(t) - s_2(t) + s_1(t))\\ 
m_4(t) &= s_4(t)-[\int_0^T s_4(t)x_{1}^{*}(t)]x_1(t)-[\int_0^T s_4(t)x_{2}^{*}(t)]x_2(t) -[\int_0^T s_4(t)x_{3}^{*}(t)]x_3(t)\\
&= s_4(t) - \frac{T}{3} x_1(t) - \frac{T}{3} x_2(t) - \frac{T}{3} x_3(t) = 0\\
\end{align*}
So the there are 3 orthonormal bases for these 4 signals.\\
\begin{align*}
x_1(t) &= 
\begin{cases}
    \frac{3}{T}, & \text{if } 0\leq t \leq \frac{T}{3}\\
    0,              & \text{otherwise}
\end{cases}\\
x_2(t) &= 
\begin{cases}
    \frac{3}{T}, & \text{if } \frac{T}{3}\leq t \leq \frac{2T}{3}\\
    0,              & \text{otherwise}
\end{cases}\\
x_3(t) &= 
\begin{cases}
    \frac{3}{T}, & \text{if } \frac{2T}{3}\leq t \leq T\\
    0,              & \text{otherwise}
\end{cases}
\end{align*}
~\\
~\\
(b)\\
In general, for $i=1,2,3,4$
$$ s_i(t) = \sum_{j=1}^{3}[\int_0^T s_i(t)x_j^*(t)]x_{j}(t)$$
Hence,
\begin{align*}
s_1(t) &= \frac{T}{3} x_1(t)\\
s_2(t) &= \frac{T}{3} x_1(t) +\frac{T}{3} x_2(t)\\
s_3(t) &= \frac{T}{3} x_2(t) + \frac{T}{3} x_3(t)\\
s_4(t) &= \frac{T}{3} x_1(t) + \frac{T}{3} x_2(t) + \frac{T}{3} x_3(t)
\end{align*}
~\\
~\\
4. (a)\\
$g(t)$ must satisfy \emph{No inter-symobol interference} property, namely, 
$$ g(jT-kT)=\begin{cases}
    1, & \text{if } j=k\\
    0,              & \text{otherwise}
\end{cases}\\$$
\emph{Nyquist criteria}, in formal, states that\\
$$ \sum_{m \in \mathbb{Z}} \hat{g}(f-m/T)rect(fT) = T \: rect(fT)$$\\
In this case, it can be simplified to
$$ \frac{1}{T} \sum_{m \in \mathbb{Z}} \hat{g}(f-m/T) = 1$$
~\\
\pagebreak
~\\
(b)\\
According to \emph{Nyquist criteria}, and $T = 1/2$
$$ \sum_{m} \hat{g}(f-2m) = \frac{1}{2}$$
Now we look into $|f|\leq 2$, since
$$\hat{g}(f) = \hat{p}(f)\hat{h}(f)\hat{q}(f)$$
When $|f|\leq 0.6$, we will have
$$\hat{q}(f) = \frac{1}{2\hat{p}(f)\hat{h}(f)}$$
Hence
$$ 
\hat{q}(f) = \begin{cases}
    1, & \text{if } |f| \leq 0.5\\
    \frac{1}{3-2|f|},              & 0.5 \leq |f| \leq 0.6
\end{cases}\\
$$
Notice that the values of $\hat{q}(f)$ in region $0.6 \leq |f| \leq 2$ can be any value because $\hat{h}(f) = 0$ there anyway.\\
~\\
(c)\\
If $\hat{h}(f)\hat{q}(f)=0$ then $\hat{g}(f)=0$, which means the \emph{Nyquist criteria} does not hold.\\
~\\
Hence, in order to avoid inter-symbol interference, \\
for each$ 0 \leq f \leq 1/(2T)$, $$\exists m \in \mathbb{Z}, \text{such that} \; \hat{h}(f-m/T)\hat{p}(f-m/T) \neq 0$$
~\\
\pagebreak
~\\
5. (a)\\
Notice that  $V_i = \int \phi_i(t)Z(t)dt$ is a Gaussian random variable.\\
Also, with Spectral density $S_Z(f)= N_0/2$, we have $$ E(V_i ^2) = \int_{-\infty}^{\infty}|\phi_i(t)|^2 S_Z(f) dt$$
Since $\{\phi_i(t)\}$ is set of orthonormal functions, we have 
$$ \int_{-\infty}^{\infty}|\phi_i(t)|^2 dt = \int_{-\infty}^{\infty}\phi_i(t) \phi_i^*(t) dt = 1$$
Therefore, we have $E(V_i^2) = N_0/2$.
Then we have $$\sigma_i^2 = E(V_i^2) - E(V_i)^2 = N_0/2 $$
Hence, we have the distribution for $V_i$
$$ V_i\sim \mathcal{N}(0, \frac{N_0}{2})$$
~\\
(b)\\
Notice that  
$$Y= \int_0^T Z(t)dt = \int_{-\infty}^{\infty} g(t)Z(t)dt $$
where
$$
g(t) = \begin{cases}
    1, & \text{if } 0 \leq t \leq T\\
    0,  & \text{otherwise}
\end{cases}\\
$$
and $Y$ is a Gaussian random variable.\\
~\\
Now, notice that
$$  E(V_i ^2) = \int_{-\infty}^{\infty}|g(t)|^2 S_Z(f) dt = \int_0^T S_Z(f) dt =  \frac{N_0}{2}T$$
Then we have
$$\sigma_Y^2 = E(Y^2) - E(Y)^2 = \frac{N_0}{2}T $$
$$ Y\sim \mathcal{N}(0, \frac{N_0}{2}T)$$
~\\

\end{spacing}
\end {document}